This thesis introduces a two-level strategy termed \acrshort{gaco}, which combines two metaheuristic algorithms, namely \gls{ga} and \gls{aco}, to grapple with both variants of \gls{idpcdu}. In the proposed algorithm, a prefiltering process and a new chromosome encoding method that decreases the chromosome length to the number of domains are also integrated. Furthermore, \acrshort{gaco} is armed with an anti-stuck mechanism that helps ants avoid being stuck, which is an inherent problem when applying this algorithm to directed graph problems. In order to analyze the proposal’s efficacy, experiments and comparisons with several algorithms on various-sized data sets were conducted. The computational results demonstrated that \acrshort{gaco} outperforms other compared algorithms in virtually all cases.

Specifically, the contributions of this thesis can be summarized as follows:
\begin{itemize}
	\item Propose a two-level algorithm combining \gls{ga} and an improved \gls{aco} to solve \gls{idpcdu}.
	\item Devise an anti-stuck strategy for ants when applying \gls{aco} to multi-domain network problems, such as \gls{idpcdu}.
	\item Conduct experiments on various test instances and comparison with several previous approaches to prove the efficiency of the proposed model.
\end{itemize}

On the other hand, the proposed algorithm still possesses several limitations. The first one to mention is the long-running time of the algorithm. The cause of this can be due to the pre-filtering process combined with the ant pathfinding. However, this is also a trade-off. Although the running time is a bit long, the results of the proposed algorithm have been proven by experiments, which outperform the compared algorithms in most cases. The second limitation is that this thesis only contains experiments in solving the \gls{idpcedu}, even though the proposed algorithm can be applied to solve both versions of this problem.

In general, \gls{idpcdu} can be applied not only in the network field but also in many other ones such as transportation and military. Therefore, it is worth researching deeper on this topic. In the future, this thesis will improve the performance of the proposed algorithm, as well as investigate more about other advanced approaches for \gls{idpcdu}.

A part of the thesis's results has been accepted at the IEEE World Congress on Computational Intelligence (WCCI 2022):

Do Tuan Anh, \textbf{Nguyen Hoang Long}, Tran Van Diep, and Huynh Thi Thanh Binh, "A Genetic Ant Colony Optimization Algorithm for Inter-domain Path Computation problem under the Domain Uniqueness constraint" in \textit{ 2022 IEEE World Congress on Computational Intelligence (WCCI)}. (accepted)
