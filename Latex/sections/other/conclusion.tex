This paper introduces a two-level strategy termed \acrshort{gaco}, which combines two metaheuristic algorithms, namely \gls{ga} and \gls{aco}, to grapple with both variants of \gls{idpcdu}. In the proposed algorithm, a prefiltering process and a new chromosome encoding method that decreases the chromosome length to the number of domains are also integrated. Furthermore, \acrshort{gaco} is armed with an anti-stuck mechanism that helps ants avoid being stuck, which is an inherent problem when applying this algorithm to directed graph problems. In order to analyze the proposal’s efficacy, experiments and comparisons with several algorithms on various-sized data sets were conducted. The computational results demonstrated that \acrshort{gaco} outperforms other compared algorithms in virtually all cases.

In general, \gls{idpcdu} can be applied not only in the network field but also in many other ones such as transportation and military. Therefore, it is worth researching deeper on this topic. In the future, we will improve the performance of the proposed algorithm, as well as investigate more about other advanced approaches for \gls{idpcdu}.
