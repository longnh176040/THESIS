\begin{center}
	{\fontsize{14}{16}\selectfont \textbf{Abstract of thesis}}
\end{center}

For the past few years, \gls{hpce} is an architecture that has been promoted to handle packet routing in multi-domain networks. However, this architecture has a potential drawback of poor scalability with
respect to the number of domains. In tackling this complicated problem, \acrfull{idpcdu}, which is employed to improve h-PCE, is focused on this thesis. The objective of \gls{idpcdu} is to find the shortest path between two given nodes that traverses every domain at most once. Since the \gls{idpcdu} belongs to NP-Hard class, this thesis introduces a two-level approach, combining the advantages of two metaheuristic algorithms, \acrfull{ga} and \acrfull{aco}. Specifically, the upper-level \gls{ga} plays the role of navigating the path for ants at the lower level \gls{aco}. Furthermore, an anti-stuck strategy that helps ants avoid being stuck is also equipped. To analyze the effectiveness of the proposed algorithm, experiments and comparisons with other algorithms are conducted. The results demonstrated that the proposed algorithm outperforms all other compared ones in most cases.

\begin{flushright}
	\begin{minipage}[t]{0.5\textwidth}
		\begin{center}
			\textit{Ha Noi}, xx May 2022\\[2cm]
		
			\textit{Nguyen Hoang Long}
		\end{center}
	\end{minipage}
\end{flushright}
\pagebreak