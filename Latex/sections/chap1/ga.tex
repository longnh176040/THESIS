\subsection {Encoding Solutions - Chromosome}
\acrfull{ga} utilizes a population of individuals, where each individual represents a candidate solution to the problem. The characteristics of an individual are represented by a chromosome. Encoding a solution is considered the most difficult step in \gls{ga}. This step requires an appropriate encoding such that each chromosome represents the corresponding solution to the real-world problem. There are several ways to represent a chromosome, for instance, a bit string.

\subsection {Fitness Function}
The fitness function, which aims to map a chromosome into a scalar value, is pondered as the most crucial component of \gls{ga}. Because each chromosome represents a potential solution, the evaluation of the fitness function measures the quality of that chromosome. Furthermore, all evolutionary operators also make use of the fitness evaluation of chromosomes. For example, selection operators use the fitness value of each individual to select parents for reproduction. When determining a fitness function, we need to consider its practicability and implementation cost.

\subsection {Initial Population}
The initialization process is the process of selecting individuals for the first population. The conventional method of forming the initial population is to choose gene values randomly from the allowed set of values. In many cases, based on available information about the search space, heuristic approaches can be applied to the initialization to take only potentially good solutions. Due to the large dimension of the search space, however, rarely all elements have a chance to be selected, which may cause premature convergence of the population to a local optimum. Therefore, the initial population's size plays an important role in performance in terms of accuracy and the time to converge.

\subsection {Selection Operators}
The selection operator takes the lead contribution to guarantee that the next generation of evolutionary progress is always better than the previous one. The new generation is generated basically through three operators, namely crossover, mutation, and elitism. In the case of crossover, the selection operator aims to choose fitter individuals to become parents. Similarly, the selection operator only nominates individuals with the lowest fitness values to be mutated in the mutation. Finally, the elitism operator retains a set of the fittest individuals to the next generation through the selection operator. In general, some of the most commonly used selection operators are roulette selection and tournament selection.

\subsection {Reproduction Operators}
Crossover and mutation are two operators in reproduction that help with producing new offspring from selected parents. In the crossover, new offspring are created generally through the combination of two parents' genetic material. The mutation, in contrast, randomly changing the values in the chromosome gene. The crossover wishes to maintain dominant materials of parents to offspring, whereas mutation introduces new genetic material into the population.

\subsection {Terminate Condition}
An ideal terminate condition for \gls{ga} is when the individuals' fitness value reaches a pre-known optimal value of the objective function. However, the randomness of this algorithm makes it extremely difficult to satisfy the above condition, and in many cases, it can be impossible due to convergence. Therefore, other more commonly used typical termination conditions include: CPU time limit exceeded; Total number of population fitness evaluations reached a specific limit; Population diversity, or in other words, the convergence of the algorithm falls below a certain threshold.