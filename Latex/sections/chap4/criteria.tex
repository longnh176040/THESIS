Table~\ref{tab:criteria} lists the criteria for evaluating the quality of the algorithms.
\bigskip
\begin{table}[htbp]
	\centering
	\caption{Criteria for evaluating the quality of algorithms}
	\scalebox{1}{
		\begin{tabular}{ll}
			\toprule
			AVG & The average function value over all runs \\
			BF & The best function value found over all runs\\
			STD & Standard Deviation \\
			RPD & Relative Percentage Differences\\
			PI & Improvement Percentage  \\
			\bottomrule
		\end{tabular}
	}
	\label{tab:criteria}
\end{table}
\bigskip

Let $S^i_{ar}$ be the solution provided by the considered algorithm $a$, for instance $i$, on the $r$-th run. Let $B^i$ be the best solution obtained among all algorithms for $i$. For minimization problems, the \gls{rpd} is calculated using Equation~\ref{equa:rpd}. The smaller the \gls{rpd} value, the better the quality of the solution found.
\begin{equation}
	\label{equa:rpd}
	{RPD}^i_{ar} = \frac{S^i_{ar} - B^i}{B^i} \times 100
\end{equation}

The \gls{pi} is also often used to compare two algorithms $a$ and $b$. Let $AVG^i_a$ and $AVG^i_b$ be the average function values given by $a$ and $b$ on $i$, respectively. The \gls{pi} of $a$ compared to $b$ on $i$ is then computed as Equation~\ref{equa:pi}. On the other hand, the greater the \gls{pi}, the larger the considered algorithm's improvement.
\begin{equation}
	\label{equa:pi}
	{PI}^i_{ab} = \frac{{AVG}^i_a - {AVG}^i_b}{{AVG}^i_b} \times 100
\end{equation}