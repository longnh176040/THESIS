To evaluate the performance of the proposed \acrshort{gaco}, 
this thesis utilize the public dataset generated in~\cite{binh2020multifactorial} for \gls{idpcdu} to evaluate the performance of the proposed \acrshort{gaco}. 

This dataset is constructed as follows. Firstly, each dataset will have three input parameters, including the number of nodes, the number of domains, and the number of edges. Secondly, an array of nodes and an array of domains are initialized separately such that the number of nodes must be greater than the number of domains. The source and destination nodes are set at the start and end of the node array, respectively. Next, from two pre-made arrays, the shortest path $p$, which is also the solution of the dataset, is created. Each edge of $p$ is assigned a weight of 1, except for the outgoing edge of the source node which is set to a weight of 2. Then the dataset is added noises to prevent simple greedy algorithms from brute-force searching for the optimal solution effortlessly. For each node in $p$, edges with random weights that lead to ones other than those in $p$ will be randomly added, especially as there is bound to be an edge with a weight of 1. Finally, the edge with a weight greater than the value of edges in $p$ is initialized on the remaining nodes of the array. This method of generating datasets ensures that $p$ is the global optimal of that dataset.

%The dataset consists of a small set with nodes ranging from 10 to 45 and a large group with 50 to 100.
This dataset comprises two sets of instances, including a small set, where the number of nodes ranges from 10 to 45 with an increment of 5, and a large one, where the number of nodes varies from 50 to 100 with an increment of 10.
The number of domains corresponding to each instance is generated by multiplying the number of nodes by a ratio of 0.5, 1.0, and 2.0. Details of the dataset are available in~\cite{idpcdu2020data}.