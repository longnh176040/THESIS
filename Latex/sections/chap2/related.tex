Notwithstanding \gls{idpcdu}'s prevalence in real-world networking, it is a relatively new problem with few existing solutions. One of the earliest efforts at tackling this problem was proposed by L.Maggi et al.~\cite{maggi2018domain}. Commencing with the idea of reducing the computational burden for parent \gls{pce} in \gls{hpce} architecture, the authors introduced a novel clustering concept that artificially decreases the number of domains without loss of optimality. Besides, a Bellman-Ford algorithm-based dynamic programming approach with a complexity of $O(|V|^22^{|D|}|D|^2)$ is given for \gls{idpcdu}. Despite its simplicity, this strategy becomes impractical when the number of domains increases, as in large-scale networks. 
Realizing this weakness, Binh et al.~\cite{binh2020multifactorial} implemented \gls{mfea}, one of the most efficient variants of \gls{ea}, to solve multiple \gls{idpcdu}s simultaneously. The feature of this proposal lies in the two-level priority-based encoding, in which the upper-level represents each node's priority, whereas the lower-level shows the index of edges that the path selects to reach the next node. Although this encoding method can provide a valid solution regardless of whether an edge exists between two particular nodes, it still possesses certain limitations. First, since both levels in this encoding are randomly generated, there is no guarantee that the resulting chromosome complies with the \gls{du}. As a result, the withdrawal of invalid chromosomes, in this case, is obligatory. Second, each chromosome's length equals the number of nodes in the graph and is doubled due to the two-level encoding. Thus, it requires a similarly complicated decoding mechanism that takes significant computer resources and computational time.

Additionally, \gls{idpcdu} has another variant, where the domain is defined on the nodes rather than on edges, coined \acrshort{idpcndu}. In simple terms, \acrshort{idpcndu} divides nodes into a fixed number of domains, while edges are no longer colored~\cite{maggi2018domain}. Anh et al. first surmounted this problem by conducting two consecutive studies in 2021. Accordingly, the first one~\cite{do2021two} proposed an encoding method that shortens the chromosome's length to the number of domains, which is significantly smaller than the number of nodes, thereby overcoming the drawbacks of the one in~\cite{binh2020multifactorial}. The second study~\cite{binh2021two} introduced a strategy to reduce the problem's search space by separating it into smaller sub-problems, in combination with a novel chromosome encoding as an order of visited domains. Even though both approaches made use of \gls{ga} and an exact algorithm, experiments illustrated that they are ineffective, probably because \gls{ga} tends to converge to a local optimum prematurely.

For an NP-Hard problem with an enormous search space like \gls{idpcdu}, some proposals hitherto often leverage the combination of two or more metaheuristic algorithms, especially \gls{ga} with \gls{aco}. This approach provides the benefits of \gls{ga}, which are the scanability of feasible solutions and avoiding premature convergence, and \gls{aco}, including the capacity to search over the subspace and then escape local optima. Therefore, the exploration and exploitation abilities of both algorithms are ameliorated. For instance, Lee et al.~\cite{lee2008genetic} developed a genetic algorithm with ant colony optimization for multiple sequence alignment, which is among the most critical and challenging tasks in computational biology. In 2020, Bang~\cite{ban2020hybridization} first presented a hybridization of \gls{aco}, \gls{ga}, and neighborhood descent with a random neighborhood ordering algorithm to combat the time-dependent traveling salesman issue. While the algorithm consumed much time, it could find better solutions than previous trajectory-based approaches in many cases, as confirmed by experiments.  

Overall, this paper designs a two-level algorithm that combines \gls{ga} with \gls{aco} to solve both variants of \gls{idpcdu}. In addition, the algorithm also includes a strategy for the problem of ants being stuck when applied to directed graphs, which is one of the difficulties with few proposals. 